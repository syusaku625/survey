\documentclass[11pt,a4paper]{jsarticle}
%
\usepackage{amsmath,amssymb}
\usepackage{bm}
\usepackage{graphicx}
\usepackage{ascmac}

%
\setlength{\textwidth}{\fullwidth}
\setlength{\textheight}{40\baselineskip}
\addtolength{\textheight}{\topskip}
\setlength{\voffset}{-0.2in}
\setlength{\topmargin}{0pt}
\setlength{\headheight}{0pt}
\setlength{\headsep}{0pt}
%
\newcommand{\divergence}{\mathrm{div}\,}  %ダイバージェンス
\newcommand{\grad}{\mathrm{grad}\,}  %グラディエント
\newcommand{\rot}{\mathrm{rot}\,}  %ローテーション
%
\title{論文調査}
\author{前田 修作}
\date{\today}
\begin{document}
\maketitle
%
%
\section{The need for mathematical modelling of spatial drug distribution within the brain Vendel(2019)}
\subsection{Introduction}
血液脳関門(BBB)は血液と脳を隔てている.BBBは、脳毛細血管の壁を構成する脳毛細血管内皮細胞によって形成されています.
隣接する脳毛細血管内皮細胞の間にはタイトジャンクションと呼ばれる多タンパク質複合体が存在し、細胞間を封鎖することで細胞間の拡散を制限している.
さらに、BBBを介した輸送は、脳毛細血管内皮細胞に存在するトランスポーターやヘルパー分子によって影響を受け、
化合物を血液から脳へ、あるいは脳から血液へ移動させる.
その結果、脳内の薬物濃度時間プロファイルは、血中濃度とは大きく異なる場合がある[1].
いったん脳に入った薬物は、拡散、脳細胞外液(ECF)のバルクフロー、細胞外交換、
脳脊髄液(CSF)のバルクフロー、代謝といった分布と排泄のプロセスを経ることになる.
さらに、薬物は特異的な結合部位(標的)と非特異的な結合部位(脳組織成分)に結合する可能性がある.
その結果、脳内の薬物濃度時間プロファイルは血中濃度とは大きく異なり[1]、また、脳内の薬物濃度時間プロファイルには局所的な差異が生じる可能性がある.
中枢神経系における薬物効果は、薬物の標的部位における濃度時間プロファイルによって決定されるため、
脳内の局所的な濃度時間プロファイルは非常に重要である.
薬物が標的となる部位に十分な濃度と時間で分布し、最適に作用して所望の効果を発揮する必要がある.
したがって、薬物の効果を予測するためには、脳内標的部位分布の定量的な把握が必要である.
しかし、ヒトの脳はサンプリングができないため、薬物の濃度-時間プロファイルの測定は非常に制限されている.
数理モデルは、脳内の薬物分布を支配するプロセスの影響を記述し、理解するのに役立つツールである.
さらに、脳内の薬物空間分布の直接測定が制限される一方で、数理モデルによって脳内の薬物の空間分布を予測することができる.
脳内薬物分布を適切に予測するためには、脳内薬物の濃度時間プロファイルを支配する上記のすべての要因を数理モデルに含める必要がある.
しかし、現在のところ、既存のモデルは、これらのプロセスのうちの1つまたはいくつかに焦点を合わせているだけである.
本総説の目的は,脳内薬物分布のモデル化に関する現状を概観し,
脳内薬物局所分布のより完全な記述を可能にする新しい手法の必要性を明らかにすることにある.
まず、脳内薬物分布に影響を与える要因についてまとめる(Factors afecting drug distribution within the brain).
次に,化合物の脳内および脳内分布に関する現在利用可能なモデルと,
これらの2つ以上の側面を統合したモデルについて概観する(「薬物の脳内局所分布に関する既存のモデル」).
最後に,「脳内薬物空間分布に関する定型数理モデルの必要性」の中で,
脳内薬物分布の予測を改善するための包括的モデルを開発するために,現在のモデルをどのように改善または組み合わせることができるかを論じている.
\subsection{Factors affecting drug distribution within the brain}
薬物の脳内分布は、薬物が標的分子と結合し、効果を発揮できる局所濃度を決定する.
脳内の薬物分布は、脳の構造的特性と薬物の構造的特性の両方が影響する.
本章では、まず脳特異的な性質と薬物特異的な性質について説明する.
次に、脳内における薬物の局所分布に影響を与えるプロセスについて説明する.
これらの過程は、脳特異的性質と薬物特異的性質の両方に依存する.
最後に,薬物分布の空間的変動が,脳内薬物濃度時間分布の空間的差異にどのようにつながるかを論じる.
\subsection{Brain‑specific properties}
脳特性は、脳の構造的な特性である.
脳内での薬物分布に最も重要な構造的特性をFig.1に示した.
血液は、脳の前部または後部から供給される動脈によって脳に供給される.
動脈は、より細い脳毛細血管に枝分かれしている.
脳毛細血管のレベルでは、血液と脳組織の間で化合物の交換が行われる.
脳毛細血管内の血液は、BBBによって脳組織と隔てられている(図1a).
脳毛細血管は再び結合して静脈となり、そこから血液は脳から心臓に運ばれる.
脳組織(脳実質)は、脳細胞外液と脳細胞から構成されている.
脳細胞外液は細胞を取り囲み、脳組織内を循環している.
CSFは、くも膜下腔、脳室、脊椎の間を循環している(図1).
血液は、血液-CSF関門(BCSFB)と血液-くも膜関門によってCSFと分離されている.
血液CSF関門は、脳毛細血管内の血液と脳室内のCSFの間に存在する(図1b).
血液クモ膜関門は、硬膜の血液とクモ膜下腔のCSFの間に位置している(図1c).
脳全体には、内因性化合物(体内由来)および外因性化合物(体外由来)の結合部位が多数存在する.
また、脳内に存在する代謝酵素が物質を化学的に変換し、新しい分子を作り出すこともある.
以下では、脳血管網、脳関門、脳組織(脳内ECFと脳細胞を含む)、髄液、脳内液の動き、結合と代謝の特性について紹介する.
\subsection{The brain vascular network}
脳には、広範な血管網が張り巡らされ、酸素と栄養を供給している(図2左).
脳表面には、酸素と栄養を脳に運ぶ太い動脈と静脈が灌流されている(図2中).
太い脳動脈は細い細動脈に分岐して脳皮質を貫通し、脳毛細血管床からなる脳微小循環に合流する(Fig.2、右).
毛細血管床を構成する脳毛細血管は、脳組織の周囲を取り囲んでいる.
老廃物は、静脈によって毛細血管床から運び出される.静脈は、血液とその中の老廃物を心臓に戻す静脈に合流する.
脳毛細血管は大きな表面積を持ち、脳組織と酸素や栄養素の交換を行う主要な場所である [2].
脳毛細血管網は非常に密であり、各神経細胞は独自の毛細血管で灌流されていると推定される[3].
ラット脳における毛細血管間の平均距離は約50μmしかありません[4-7].
脳毛細血管は,次節で説明する脳関門によって脳から分離されている.
\subsection{The barriers of the brain}
脳毛細血管内の血液と脳を隔てる3つのバリアが知られている.
\begin{enumerate}
    \item 脳毛細血管内の血液と、脳内ECFや脳細胞などの脳組織とを隔てているBBB. 
    \item 脳毛細血管内の血液と脳室内の髄液を分離する「BCSFB」.
    \item 硬膜の血管内の血液とクモ膜下腔の髄液を分離する血液クモ膜関門(図1参照).
\end{enumerate}
\subsection{BBB}
BBBは、毒性物質や有害物質の流入から脳を保護している [8].
さらに、イオン、分子、白血球の脳への出入りを制御することで、脳の恒常性維持に役立っている [9].
BBBは血液と脳を隔てており、脳毛細血管の壁を構成している脳内皮細胞からなる.
薬物によっては、BBBを越えて輸送することが困難な場合があります.
通常、脳内皮細胞は固く閉じた細胞層を形成しています[10](Fig.3a).
タイトジャンクション、脳内皮細胞間の狭い空間に存在する多タンパク質複合体、
および隣接する脳毛細血管内皮細胞間のフェネストレーション(小孔)の欠如が、化合物が細胞間隙を通過することを困難にしている [11].
脳内皮細胞の周囲には、アストロサイト(支持細胞、「脳組織と髄液」参照)が神経細胞と周皮細胞と接続しており、
後者はBBBの機能を調節している [8]。これらは共に、いわゆる神経血管単位を形成しており、これが実際の脳のバリアとなっている.
\subsection{BCSFB}
脳毛細血管内の血液と脳脊髄液を分離しているのがBCSFB。脳を正常に機能させるために、化合物の交換を調節している.
脳室内にある脈絡叢の上皮細胞で構成されている(図1).
これらの細胞は、タイトジャンクションによって強く結合している(図3b).
一方、BCSFBの脳毛細血管は、BBBのそれとは異なり、柵状(孔を持つ)であり、非常に透過性が高い.
\subsection{The blood–arachnoid barrier}
血液くも膜関門は、硬膜にある(柵状の)脳毛細血管と、くも膜下腔にある髄液を隔てている(図1参照)[12 14].
この関門は、クモ膜細胞(硬膜とクモ膜下腔の間にある上皮細胞)がタイトジャンクションで結合した層で形成されている(図3c).
\subsection{The brain tissue and the CSF}
脳組織は、脳内ECFと、細胞内液(ICF)を含む細胞からなる.
脳組織は、脳血管系(「脳血管網」も参照)で灌流され、CSFに包まれている.
以下、脳内液、脳細胞、脳脊髄液の性質について説明する.
\subsection{The brain ECF}
脳ECFは脳細胞を取り囲み、脳組織の約20\%を占めている.
血漿との混同を避けるため、脳間質液とも呼ばれるが、実はこれもECFの一つである.
脳内ECFは内因性及び外因性化合物の輸送に極めて重要である [15, 16].
脳内ECFは、BBBを構成する脳毛細血管壁で血漿がろ過されることにより生成される.
タンパク質はBBBを通過できないため、脳内ECFの組成は血漿のそれと似ているが、最小限のタンパク質しか含まれていない.
\subsection{The brain cells}
脳細胞は、神経細胞、支持細胞(グリア細胞)、周皮細胞に分類される.
神経細胞は、電気的および化学的なインパルスによって情報を伝達する興奮性の脳細胞である.
1本の長い軸索と、1本または複数の短い樹状突起が細胞体に付着した典型的な形態をしている.
複数の軸索が密集している神経路と呼ばれる構造もある.
グリア細胞は、神経細胞を支持・保護するもので、アストロサイト、オリゴデンドロサイト、ミクログリアなどがある[17].
このうちアストロサイトは、神経細胞の活動に酸素と栄養の輸送を合わせるために、局所的な血流を調節する重要な機能を持つ [17-21].
アストロサイトは、脳内皮細胞と神経細胞の両方と接触している.
最後に、周皮細胞は脳内皮細胞を取り囲み、収縮運動によってBBBの透過性と脳毛細血管の血流を調節するのに役立っています[17].
これらの脳細胞を合わせると、脳組織の体積のほぼ80\%を占める[17].
脳の深部にある白質と、より表層にある灰白質では、細胞構成が異なる.
白質は、神経細胞から枝分かれした髄鞘のある長い軸索を含む神経路が大部分を占めている.
髄鞘とは、軸索がミエリンで覆われ、神経路の情報伝達が速くなることです.
灰白質は、神経細胞(細胞体、樹状突起、髄鞘のない軸索)、グリア細胞、脳毛細血管から構成されています.
\subsection{The CSF}
髄液は4つの脳室に存在し とクモ膜下腔に存在します.
髄液は、脳を機械的に保護し(衝撃や傷害から)、老廃物の排出を助け、心臓の拍動に伴う脳の血液量の変化を補う役割を担っている[22].
髄液は主に脳室の脈絡叢の上皮細胞によって産生される[23](Fig.1).
最近では、脳毛細血管壁を越えて脳ECFに液体が濾過される結果、脳CSF循環全体の中でCSFが生成されるという仮説が提唱されている[24].
髄液は、脳内ECFと同様の組成で、タンパク質濃度の低い透明な液体です。であり、脳内ECFと同様の組成である.
\subsection{Fluid movements within the brain}
脳組織内の恒常性を維持するためには、脳内ECFの定期的なリサイクルとクリアランスが必要である [25].
脳室内の脳外液と髄液の間にある上衣細胞層(図3b参照)とクモ膜下腔内の脳外液と髄液の間にある小膜細胞層(図3c参照)は、
いずれも比較的透過性が高いため、脳外液と髄液の間を自由に循環することができる.
したがって、脳内ECFとCSFの間を液体が自由に循環している[26, 27].
脳内ECFとCSFの両者の動きについては後述する.
\subsection{Brain ECF movement}
脳内ECFは、脳毛細血管内皮壁からの体液の分泌により生成される.
これは、イオン勾配に反応してBBBを通過する水の受動的な動きから生じる [25].
脳内では、脳内ECFは脳内ECFバルクフローによって細胞外空間内を移動する.
脳内ECFバルクフローは、静水圧 [27, 28] または脳動脈の脈動運動 [29] によって駆動される.
脳ECFバルクフローは、脳室およびクモ膜下腔のCSFに向けられる.
そこでは、CSFはその回転のためにシンクとして機能する(「CSFの動き」を参照) [30].
あるいは、脳内ECFは毛細血管や動脈壁を越えて直接リンパ系に排出されることもある [30].
拡散に対する脳内ECFバルクフローの重要性については、これまで議論がなされてきた [27, 31, 32].
最近提唱された "glymphatic mechanism "は、
グリア細胞によって制御される脳ECFを介した傍動脈腔から傍静脈腔への対流的な体液輸送を説明する[23, 29, 33, 34].
この「グリムパティック機構」は、グリア細胞に依存し、脳外のリンパ系による老廃物の除去に似ていることから、その名がついた [35, 36].
脳内ECFは動脈または細動脈から脳内ECFに入り、脳内ECFは静脈または細静脈に沿って出ていく、脳内ECFとCSFの間の体液交換を伴う[33, 36].
この体液交換は、アストロサイトのエンドフィードに位置し、障壁を越えて水の輸送を促進する、
いわゆるアクアポリン-4チャンネルに依存すると示唆されている[33, 35-37].
グリムパスのメカニズムはメカニズム的な根拠を欠いているため、数学的モデリングが利用される.
脳ECFバルク流の「glymphaticメカニズム」を考慮した最近のモデリング研究では、脳ECF内の輸送は拡散が支配的であることが示されている [38, 39].
拡散に対する脳内ECFバルクフローの重要性については議論があるが、脳内ECFバルクフローがてんかんを含む脳疾患に影響を与えるという証拠がある[25].
\subsection{CSF movement}
髄液は、脳脊髄液膜を構成する脈絡叢の上皮細胞により産生される(図1b、3b).
髄液は一般に脳室とクモ膜下腔の間を循環した後、末梢血流の血液に再吸収されると考えられている(Fig.1c).
CSFはまた、リンパ系に吸収されることもある [40].
髄液の一部は、Virchow-Robin腔(脳組織を貫通する血管の周囲の液体で満たされた管)または傍動脈腔を介して脳組織に吸収されうる [25, 31, 41-43].
Virchow-Robin腔は、脳から老廃物を排出する経路として機能しており、脳と(全身)免疫系の相互作用の場でもあるという証拠がある [23].
CSFはCSFシステム全体で生成されると考え、CSF循環をより複雑なものとして記述する新しい見解が出始めている[23-25].
そこでは、CSF循環には、CSFバルクフロー、脳ECFとCSF間の液の脈動的な往復運動、
脳ECFとCSF間のBBBと細胞層を介した液の連続的な双方向交換が含まれている(図3参照)[23].

\subsection{Existing models on the local distribution of drugsin the brain}
薬物が脳内にどのように分布しているかを理解することは、薬物の効果を正確に予測する上で非常に重要である.
しかし、脳内における薬物の分布については、まだ多くのことが分かっていません.
数理モデリングは、実験だけでは得がたい、あるいは不可能な情報を提供することができます.
それによって、モデルは研究中のメカニズムに対する洞察を得るのに役立つ.
次のサブセクションでは,脳内薬物分布に関する既存のモデル(関連するプロセス)をレビューしている.
最初の2つのサブセクションでは,脳毛細血管系を介した薬物輸送のモデル("Modelling drug transport through the brain capillary system")
とBBBを越えた輸送のモデル("Modelling drug transport across the BBB")について述べている.
次の4つのサブセクションでは、脳内薬物分布と脳外薬物排出に関するモデルを説明する.
脳ECF内の薬物分布("Modelling drug transport within the brain ECF")、
細胞内交換("Modelling intraxtracellular exchange")、薬物結合速度("Modelling drug binding kinetics" )
と薬物代謝("Modelling drug metabolism in the brain")が挙げられる.
各過程に関連するパラメータの値域と単位は、ラットとヒトについて付録で示した.
脳の部位や薬物の状態を表す複数のコンパートメント間の交換に関するモデルは、「コンパートメント間の薬物交換のモデル化」で扱っています.
その概要はTable 2に示されている.最後に,「モデル特性の統合」では,脳内薬物分布の数学的記述を統合した脳内薬物分布モデルの現状を概観している.
その概要をTable 3に示す.
\subsection{Modelling drug transport through the brain capillary system}
ここでは、特に脳への薬物輸送、ひいては脳毛細血管網による薬物送達に焦点を当てたモデルについてのみ言及する.
毛細血管から脳などの組織への化合物の分布は、Krogシリンダーで表すことができる.
クローグシリンダーは、組織を、中心に1本の毛細血管を持つ円筒として表現する[110].
このモデルはよく確立されており、脳を含む広範な組織への酸素や他の分子の供給を説明するために広く用いられてきた[111].
クローグシリンダーの例を図9に示すが,ここでは,脳毛細血管が脳組織の層に囲まれている[111].
そこでは,脳は4つのサブユニットで表され,$S_j(1≤j≤4)$で示される.
薬物の拡散フラックスは,$\phi_0$ で示される脳毛細血管と脳組織の間,および $\phi_j(1≤j ≤ 4)$で示される脳組織のサブユニットの間で発生する.
Kroghシリンダーは,BBBを介した単純な受動的薬物輸送に対する脳毛細血管の血流の影響を調べるために用いることができる.
脳への薬物受動輸送の速度定数$k_in$は,脳毛細血管の血流量Qおよび脳内に抽出される化合物の割合$E$とRenkin-Croneの式で関連付けられる[112, 113].
\begin{equation}
    k_{in}=\frac{QE}{V_{brain}}\quad with\ E=1-e^{\frac{-PS}{Q}} \quad or\ E=\frac{C_{in}-C_{out}}{C_{in}}
\end{equation}
$V_{brain}$は脳の体積、$E$は化合物抽出比、$PS$(m s-1 m2)はBBB透過性表面積積、
$C_{in}$(mol L-1)は脳毛細管に入る薬物濃度、$C_{out}$(mol L-1)は脳毛細管から出る薬物濃度とする.
式(1)より、BBBを容易に通過する薬物(PSが高い)の場合、血漿から脳ECFへの薬物の抽出は、脳毛細血管の血流量によって制限されることがわかる。
BBBを通過しにくい(PSが低い)薬物の場合、血漿から脳内ECFへの薬物抽出はBBBの透過性により制限される.

Kroghシリンダーは単一セグメントに限定されており、関門に沿った拡散は考慮されていない.
PSは生理的な定数であるという前提で駆動するが,実際には脳毛細血管の血流量や半径に大きく依存するため,生理的な透過率と同一ではない[114].
近年,脳血管網の大規模な解剖学的モデルが開発されている.
そこでは,医用画像のセグメンテーション[115-119]や幾何学的な構築[120-124]に基づいて,脳血管網全体が構築されている.
これらのネットワークは、多数の血管セグメントがノードで接続されたものであり、
ネットワークを定義するパラメータ(血管の半径、体積、長さなど)は、画像、実験データ、ランダム分布に基づくものである.
これらの脳血管ネットワークは,ドラッグデリバリーに応用することができる[116, 125].
脳腫瘍へのドラッグデリバリーに関するモデルでは,画像に基づく脳毛細血管網が,脳組織の立方体メッシュ表現に結合されている[116].
そこでは,微分方程式系が,血管内の薬物輸送,組織への(受動)薬物輸送,組織内の薬物拡散と崩壊を記述している.
最近の数学モデルは,脳毛細管による脳への薬物送達と,それに続くBBBを介した能動輸送を記述している[125](Fig. 10, left).
このネットワークでは,各脳毛細血管がそれぞれの体積の脳組織に供給される.著者らは,BBBを介した受動輸送は考慮していない.
このモデルでは,脳毛細血管のネットワークは,立方格子の一定のトポロジーで記述される.
立方格子のネットワーク全体は、体積1cm3の脳組織の一部を表している.
ネットワーク内の脳組織の格子の体積は同一であり、脳内の空間差は考慮されていない.
一定濃度の薬物が左の表面($x = 0$)からネットワークに入る.
全体の血流はネットワークの左側から右側($x = 0$から$x = 1$)へ向けられる(Fig.10(右)参照).
BBBを通過する薬物の輸送速度は、BBBの透過性と脳毛細血管の血流速度に影響されるが、
BBBを通過する薬物の量は、血漿タンパク質との薬物結合に影響される.
薬物が血漿タンパク質と結合すると、脳を通過できる未結合薬物の濃度が減少する.
しかし、血漿タンパク質との薬物結合を考慮したモデリング研究はごくわずかである.
一例として、化学療法薬であるドキソルビシンの血漿タンパク質に対する高い親和性は、
薬物濃度を遊離と血漿タンパク質結合に分割することによって説明されている[126].
\subsection{Modelling drug transport across the BBB}
ほとんどの薬物は血液から脳に入るため、BBBを通過する必要がある.
したがって、脳内の薬物分布に関するモデルにBBBを含めることが重要です.
BBBを通過する受動輸送と能動輸送では、異なるモデリングアプローチが必要です.以下では、これらについて説明します.
\subsection{Passive BBB transport}
BBBを介した薬物輸送は、しばしば脳内ECFからの化合物の損失、すなわち脳内ECFから血漿への一方向の不可逆的輸送と表現される[107, 127-132].
しかし、BBBを介した受動輸送は双方向性であり、薬物は血液から脳内ECFへ、脳内ECFから血液へ輸送される.
以下では,BBBの受動輸送を定量化するいくつかの方法について説明する.
血漿と脳内ECF間のBBBを通過する薬物の受動的フラックス$φ$は双方向性でBBBに垂直である.
BBBの透過性と血漿と脳ECF間の薬物濃度差に依存する。これは以下のように定義できる[111, 126, 133].
\begin{equation}
    \phi_{pas}=P(C_{pl}-C_{ECF})
\end{equation}
$\phi_{pas}$はBBBの単位面積あたりの薬物の双方向受動流量、$P(m s^{-1})$はBBBの薬物透過性、
$C_{pl}(mol m^{-3})$は血漿中の薬物濃度、$C_{ECF}(mol m^{-3})$は脳ECF中の薬物濃度とする.
BBBを介した双方向の単純な受動的薬物輸送の結果としての脳内ECFにおける薬物濃度の変化は、
速度定数 [129, 132, 134-136] またはトランスファークリアランスパラメータ [137-142] を用いて記述することが可能である.
\begin{equation}
    \frac{dC_{ECF}}{dt}=k_{BBB}(C_{pl}-C_{ECF})\\
    V_{ECF}\frac{dC_{ECF}}{dt}=CL_{BBB}(C_{pl}-C_{ECF})\\
    with\ C_{ECF}=\frac{A_{ECF}}{V_{ECF}}
\end{equation}
ここで、$k_{BBB} (s^{-1})$ は BBB を通過する薬物輸送の速度定数、
$CL_{BBB} (m3 s^{-1})$ は BBB を通過する薬物輸送の移動クリアランス、
$A_{ECF} (mol)$ は脳 ECF 中の薬物のモル量、
$V_{ECF} (m^3)$ は脳 ECF の体積である.
一部の研究では、脳ECF中の薬物量ではなく、脳組織(脳ECFと脳ICFを含む)中の薬物量をモデル化しており、
すなわち$A_{ECF}$ではなく$A_{brain}$が用いられている[134, 136].
BBBを通過する薬物の受動的フラックスは、式(2)で定義され、
経細胞輸送による受動的フラックスと受動的な傍細胞輸送による受動的フラックスの和である.
したがって、受動的透過性Pは、以下の式で与えられる[143].
\begin{equation}
    P=P_{trans}+\frac{D_{para}}{W_{TJ}}
\end{equation}
ここで、$P_{trans}(m s^{-1})$は受動的な細胞外透過性である.
$D_{para}(m^2 s^{-1})$はBBB細胞間隙を通過する薬物の拡散係数、$W_{TJ}(m)$は細胞間隙の幅を表す.
式(4)は、タイトジャンクションの幅がBBB細胞間隙を拡散する薬物の移動距離と等しいという仮定に基づくものである.
しかし、電子顕微鏡による観察では、BBBのタイトジャンクションは蛇行した形状をしており [144] 、
したがってWTJは拡散する薬剤の実際の移動距離を過小評価する可能性が高い.
細胞外拡散はBBBの総表面積の0.006\%でしか起こらない[143].
したがって、受動的な傍細胞性輸送と受動的な経細胞性輸送の相対的寄与を考慮した補正係数(BBB表面積分率)を使用する必要がある[143].
BBBの両側の非撹拌水層を介した輸送は、
BBBの細胞の頂膜(血液側)と腹膜(脳側)の両方を介した化合物の輸送を広範囲に記述する最近のモデルに含まれている[54].
膜の両側において,BBB,脳ECFおよび非攪拌水層内の化合物濃度に対する受動的な細胞外透過性,
細胞外輸送,能動的透過性および非攪拌水層の影響が記述されている(図11).
\subsection{Active BBB transport}
能動輸送は、分子を濃度勾配に逆らって膜の一方から他方へ移動させるため、エネルギーを必要とする.
能動輸送は一方向の輸送であり、能動輸送タンパク質を介するため、受動輸送の場合とは別の記述が必要となる.
以下、能動的BBB輸送の定量化のいくつかの方法について説明する.
最も単純な形として、受動輸送と能動輸送の両方によるBBBを越える総フラックス
($\phi_{tot}$)を受動透過率と同じ方法で記述し(式(2))、
それによって能動輸送の一方向性および能動輸送タンパク質の飽和を無視することである.
\begin{equation}
    \phi_{tot}=P_{tot}(C_{pl}-C_{ECF}) \\
    \phi_{tot}=PAF_{in}(C_{pl})-PAF_{out}(C_{ECF})
\end{equation}
ここで,$P_{tot}(s-1)$はBBBを通過する全(受動+能動)輸送速度,
$AF_{in}$は薬物の脳への能動輸送に対する親和性 [141],
$A_F_{out}$は薬物の脳からの能動輸送に対する親和性 [141],である.
全透過率$P_{tot}$は、受動的BBB透過率$P$に血液脳分配係数を乗じた積として記述されることが多い[1,138,139,141,142,145].
あるいは、BBBからの薬物の能動的な輸送は、能動的な透過率$P_{act}$によって記述することができる[54].
この活性透過率、$P_{act}$は、$P_{act}$が$P_{P-gp}$および$P_{BCRP}$の$P_{BCRP}$を含む
個々のトランスポーターによる活性BBB輸送の合計に等しいように、
特定のトランスポーターに特異的であることができる(「脳関門を越える薬物輸送」参照)[54].
式(5)の総フラックス$\phi_{tot}$の記述は、傍細胞輸送がある場合には、化合物が細胞を迂回し、
細胞上の活性トランスポーターと相互作用しないため、成立しない[54].
その場合、式(2)を用いる必要がある.活性輸送は、本来、酵素変換を記述するために用いられるMichaelis-Menten動力学に従って働くと一般に仮定される.
このようにして、BBBを通過した薬物の脳内または脳外への能動クリアランス、$CL_{act}$は以下のようにモデル化される[1, 127, 138, 139, 146, 147].
\begin{equation}
    CL_{act}=\frac{T_m}{K_m+C}
\end{equation}
$Tm(μmol L^{-1}s^{-1})$ はBBBを通過する薬物の最大輸送速度(外向き輸送では負)、
$K_m ( μmol L^{-1})$ は$T_m$の半分に達する遊離薬物の濃度、
$C(μmol L^{-1})$ は血漿中濃度$C_{pl}$(活発な内向き輸送の場合)または脳ECF中濃度$C_{ECF}$(活発な外向き輸送の場合)であるとした.
\end{document}